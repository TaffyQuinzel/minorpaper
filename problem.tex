\chapter{The Problem}
For a programming language to be usable and good it needs to be tested on  the following criteria\cite{khedker1997makes,graham2004hackers}
\begin{enumerate}
  \item Writability and Readability
  \item Simplicity
  \item Definiteness
  \item Orthogonality
  \item Expressiveness
  \item Implementability
  \item Efficiency
  \item Libraries
  \item Time
  \item Hackability
  \item Succintness
  \item Redesign
  \item External Factors
\end{enumerate}

Let's touch on each of these briefly, before we connect them to MC.
\paragraph{Writability and Readability}
of a programming language determines how good the connection between man and machine is. The programmer must be able to write programs he understands easily, even after months or years of not looking at it. This makes debugging much easier.

Documentation is makes up the basic part. But to really be able to make a program reabable, even after a prolonged absence, the language needs to be it's own documentation. This can be done via a manner of ways:
\begin{itemize}
  \item Keywords
  \item Abbreviations and concise notation
  \item Comments
  \item Layout or format of programs
  \item No overuse of notatio
\end{itemize}
These points will be explained later on with examples off MC.\cite{}

\paragraph{Simplicity}
of a language is decided by it's features. These should be easy to learn and remember. To make this more provable we will make use of these subpoints:
\begin{itemize}
  \item Structure
  \item Number of features
    \item 
\end{itemize}
\paragraph{Definiteness}
\paragraph{Orthogonality}
\paragraph{Expressiveness}
\paragraph{Implementability}
\paragraph{Efficiency}

\paragraph{Libraries}
are a necessity for every programming language. Without them the language would be useless and the programmer would have to build every feature from the ground up. This makes programming in the language take a long time.

\paragraph{Time}
is something everything needs to build momentum and a stable userbase, even programming languages can't excape this. For now we won't be able to really test this, as MC is still in it's development stages. For this reason we will leave this out.
\paragraph{Hackability}
is the ability to bind the language to one's will or to form the language to one's needs. 
\paragraph{Succintness}
\paragraph{Redesign}
enables the evolution of a program written in the language. This makes it easier to go from a rough prototype to a fully featured program.
\paragraph{External factors}
play a big role in the adaptation of a language. Without some big 

Do keep in mind that most of these criteria can be subjective to the user of the language and are not 100\% provable. We will try to discuss the most objective parts for provability's sake.

First I will explain MC in more detail, after which it can be put to the test. I will see how far it already has the above described criteria and in which areas it still needs some improvement.

  

MC in it's basic form is not

create a standard framework which makes the language usefull

The language in it's basic form is not very use

%% google for what makes a programming language usefull
