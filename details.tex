\section{Details}
We will now look at MC itself and explain the syntax and the built-in features by looking at examples from the standard library.

But first we will look at the basics of MC.

\subsection{The basics}
let's start with giving a simple example of some MC source code:
%insert first picture from jarno

Let's go over the different parts:

\paragraph{declarations} enable the user to define their own keywords.
The keywords are in parenthesis.
The arrows make up the order of the declaration.
As shown above it is possible to have functions with variables on the left side of the declaration.
the variable after the final arrow is the return value of the function.

\paragraph{function definitions} occur after the declarations.
The rule starts the input and output below the line of dashes.
On the left of the arrow the input is stated and on the right the output.
Above the line are where the function actually does the work.
It tests the premises or conditionals to see if the rule can be executed.
You can define multiple rules per function and they will be simultaniously executed.
The program will split and rules that don't match will stop.

\paragraph{Data and Func} are both used to declare. Func defines a function and Data defines a datatype.
The major difference between them is that Data is two-way traverseble and Func only one-way.
This means that when you have a variable which has a type created by Data you can extract the start values.
You cannot do this with Func.
Func also has the ability to curry.

\paragraph{} Both however define terms by using types.\footnote{\cite{pierce2002types}}

\subsubsection{Going higher}
Take a look at the following source code:
%insert second picture from jarno

\paragraph{} here we see a similarity with the previous code example.
There are again declarations and rules.
This time however we also see a declaration in a rule.
Let's go over the new keywords one by one.

\paragraph{TypeFunc} is similiar to Func in that it declares samething.
However, as the name suggests, TypeFunc defines a type and not a term.
TypeFunc uses kinds to define types.\footnote{\cite{pierce2002types}}
These types can than be implemented by funcs as you can see in the example.

Another difference are the arrows used.
Intsead of using the normal arrow, ->, the => arrow is used in TypeFunc declarations.
This is to add clarity as to which Func is being declarated.

\paragraph{Module} is used to create a collection of declarations.
In this case two Func declarations are used.
You can also use the TypeFunc or Data declarations within a module.

When we take the example module Number, we can see it takes an argument.
As we can see in the Func declarations it is implied that this argument is a type and not a variable.
These variables are called type-variables and are destinguishable by a apostrophe as a start of their name.

The declaration of "add" and "sub" within Int assume that the + and - operators are already defined.
We will later see how to use built-in and system operators.\cite{something}
Because Int makes use of the module Number by feeding it a type-variable, there is no need for declarations of "add" and "sub."
They exist within the module Number and can thereby be used by any definition of the Number.





%USE THIS SOMEWHERE LATER



%\subsubsection{going without definition}
%In the following source code we something new, namely a lambda:
%insert third picture of oarno

%$\left.\begin{minipage}{5cm}
%\begin{lstlisting}
%(\ x ->
  %x + 4 -> res
  %res -2 -> res'
  %res') -> lam
%\end{lstlisting}
%\end{minipage}\right\rbrace$ test drie
%\begin{lstlisting}
  %----------------
  %bar -> lam
%\end{lstlisting}


%\paragraph{Lambdas} are defined by:

%\begin{lstlisting}
%(\ variables ->
  %do stuff) -> input\_for\_lambda
%\end{lstlisting}


\subsection{Standard Library}
Now let's have a look at what is in the standard library at this time and explain some more funtionalities of MC.
The standard library consists of the following:
\begin{multicols}{2}
\begin{enumerate}
	\item StandardLibrary
		\begin{enumerate}
			\item Boolean
			\item Monad
			\item TryableMonad
			\item Number
			\item Newnumber
			\item Match
			\item Prelude
			\item Record
		\end{enumerate}
	\item BasicMonads
		\begin{enumerate}
			\item either
			\item id
			\item list
			\item option
			\item result
			\item state
		\end{enumerate}
\end{enumerate}
\end{multicols}

\subsubsection{system-types}
When we look at boolean we see how system-types can be implemented within MC:
\begin{lstlisting}
TypeFunc "Boolean" => Module
Boolean => Module {
  Func "True" -> Boolean^system
  Func "False" -> Boolean^system
}
\end{lstlisting}

First you specify which system-type you want to implement after which you tell MC it is a system-type.
Using this manner of loading system-types you don't have to write everything from scratch and quickly build upward.

Ofcourse this only tells MC that true and false are of type boolean.
MC still doesn't undrestand which value they have.
For this we simply implement the Boolean Module, as is done in prelude:

\begin{lstlisting}
import boolean

Boolean => Boolean {
  True -> TrueBoolean^system
  False -> FalseBoolean^system
}
\end{lstlisting}

First you import the Boolean Module from the file boolean.
Then you implement Boolean and tell which term True and False will get.
Because of scoping you con use the same name for Boolean as the Module name for Boolean.

\paragraph{} When we look at Monad and TryableMonad we see a few new keywords aswell as a more complex practical aplication af MC.

\subsubsection{Monad}

\begin{lstlisting}
import prelude

TypeFunc "Monad" => (* => *) => Module
Monad 'M => Module {
  ArrowFunc 'M 'a -> ">>=" -> ('a -> 'M 'b) -> 'M 'b   #> 10 L
  Func "return" -> 'a -> 'M 'a

  Func "MCons" -> *
  MCons -> 'M

  Func "returnFrom" -> 'a -> 'a
  returnFrom a -> a

  Func "lift" -> ('a -> 'b ) -> 'M 'a -> 'M 'b
  a >>= a'
  --
  lift f a -> return(f a')

  Func "lift2" -> ('a -> 'b -> 'c) -> 'M 'a -> 'M 'b -> 'M 'c
  a >>= a'
  b >>= b'
  --
  lift2 f a b -> return(f a' b')

  TypeFunc "liftM" => (* => *) => * => *
  N >>= a
  f a -> b
  lift^N(return^N b) >>= res
  --
  liftM f N -> return res
}
\end{lstlisting}

\paragraph{}
In Monad the first thing we see is a TypeFunc which tells MC what exactly a monad is.
It takes as an argument "( * => * )" and returns a Module.
As we have explained in \cite{sectionstuff} a Module is used as a container.
With this in mind a Monad is nothing but a collection of declarations which takes an argument.

When we look at the argument it takes we notice it is a function of some sort which takes a type and returns a type.
Which in kinds are described as *.

The 'M stands for a monad it takes.
This might be confusing if you have any knowledge about monads, since monads are not always a function.
That's because Monad is actually a basis for monad transformers.
As we will see in \cite{sectionstuff}, it comes in very handy when working with monads.

But let's first look at the new keywords we see in Monad.

\paragraph{Import} does what most programmers would expect, it imports the functions from the file specified after "import."

\paragraph{ArrowFunc} is an abreviation of Func.
It creates a function which can take arguments placed on the right of the "operator", from the line below it.

It gives an error if the number of right arguments is not met.

At the end of ArrowFunc we see another type of arrow, \#>, the priority arrow.
It gives the possibility to give it a priority and say if it is left- or right associative.
ArrowFunc is standard right associativity, so there is no need to specify that other than clarity.

\paragraph{comments} are a simple but essential feature.
They enable the programmer to explain what is happening when the code is complex.
In MC a single comment line starts with \$\$ and when we want to comment a block we use \$* to start and *\$ to end the block.


\subsubsection{TryableMonad}

\begin{lstlisting}
import prelude
import monad

TypeFunc "TryableMonad" => ( * => * ) => Module
TryableMonad 'M => Monad(MCons^M) {
  inherit 'M

  Func "try" ('a -> MCons^'M 'b) => ('e -> MCons^'M 'b) => MCons^'M 'a => MCons^'M 'b

  Data "e" -> String

  $$ return the monad of the tryable monad, this way you can use the tryable monad as a the normal monad
  Func "getMonad" -> MCons^'M
  getMonad -> 'M

  $$ return the tryable monad, this way you can use the tryable monad
  Func "Tryable" -> 'a -> 'a
  Tryable -> a -> a
}
\end{lstlisting}

Now let's look at what's new in TryableMonad

\paragraph{Inherit} has a similiar functionality as import, but it imports the funtions of a variable.
Note that this is only possible if the variable is based on a module, as only a module can contain finction definitions and declarations.


\subsection{monads implementated}

When we look at a basic implementation of the monads in MC, we will see how using monadtransformers is much easier in the end.

We will first take a look at the ID monadtransformer and then see how this will make the transformers usable.

\begin{lstlisting}
import prelude
import monad

TypeAlias "Id" => * => *
Id 'a => 'a

TypeFunc "id" => Monad

'M >>= b
-----------------
id 'M a => Monad(Id) {
  inherit b

  x >>= k -> k x
  return x -> x
}
\end{lstlisting}

\paragraph{TypeAlias} creates a datatype on kind level.
This way we can use Id as a signature to implement the "id" monad.
Without TypeAlias there would be no way of actually implementing a monad.

TODO: WHY DOES ID HAVE A BIND THAT IT INHERITS AND NOT JUST 'M DIRECTLY

\paragraph{} As we can see, having a basic understanding of MC, the id monad does nothing with the input.
This makes it pass the functionality of the input monadtransformer directly as output, thus creating the monad itself.

Without having an id monad there would be no way of using the monad transformers, making them pointless.


\subsection{put to the test}
Now that we have a better understanding of MC, we can put it to the test using the criteria described in \cite{sectionstuff}

\begin{multicols}{2}

\subsubsection{Read- \& Writability}
\paragraph{Keywords}
are few and clear in what they do.
furthermore you can define your own keywords by using the different declaration methods.\cite{sectionstuff}
Which means you can basically build your own language which is stronfly typed and tuned to your needs.

\paragraph{Abbreviations and concise notation}
is quite clear in MC.
it has almost no sharthand functionality, which makes it clear what is what.
The notation is alsa quite clear and kept to a minimum, which does exactly what it should and nothing more.
it is a quick way to recognize the code structure and it's meaning.

\paragraph{Comments}
are implemented in a proper way as shown in \cite{sectionstuff}.

\paragraph{Layout or format of programs}
is mostly a fixed layout, like most modern functional programming languages.
This is a price you pay for having a strong type system.

\paragraph{No overuse of notation}
rarely comes into play.
at most you will see the bars fairly often and declarations quite often.

\subsubsection{Simplicity}
\paragraph{Structure}
is one of the stronger points of MC.
Declarations and there implementation make it so there is a good structure to be made in the source code.
We can group the declarations together and the implementations or group the declarations together with their implementations.

\paragraph{The number of features}
are basic but very expandable.
This makes it easy to understand even complex programs, when you see how they are build on top of the basics.

\paragraph{Multiple ways of specification}
are achievable through declaring them.
Though the basics of the language is quite strict, we can build almost anything with them.
So if we want we can declare multiple ways of doing the same thing.

\paragraph{Multiple ways of expressing}
are achievable by overloading the standard operators or declared operators.
This can make it very obscure what is actually done, but gives us the freedom to create and tune our own operators.

\subsubsection{Definiteness}
is definitly there, but not completely realised yet.
Atleast the parser monad should be implemented in the standard library.
And to be really sure an example compiler would be handy or some proper documentation.

All the basics are there to realize the goal of MC, it just needs to be implemented.

\subsubsection{Orthogonality}
is probably one of the things MC aims to achieve the hardest with it's type system.
This has definitly been achieved.
The strong type system which makes use of three levels: terms, types and kinds.
Makes sure the language is similar in predictability as languages such as Ada.

\subsubsection{Expressiveness}
is quite high because of the high level of abstraction in MC.
This enables you to express very complex ideas with very little clear code.

\subsubsection{Efficiency}
is something we can't really test without a fully functional compiler.
For we cannot test this.
For more on this subject see \cite{jarnoshizzle}

\subsubsection{Libraries} are definitly lacking at this moment in time.
On the upside, there is a link to system stuff.
FIX THIS PLEASE

\subsubsection{Time} has done quite alot for MC, when you look at the development.
Other than that there is still alot that can happen in time.
Only time will tell.
THIS IS CRAP

\subsubsection{Hackability} is a big part of MC, you can create the language you want.
MC itself can be quite powerfull by itself, but it really shines when you build a compiler for the language you had in mind.

This might be quite a big step for alot of programmers, but it can be very productive if they try it.
It can save them time in the end.

\subsubsection{Succinctness} is something MC falls short.
You have to declare every function you'll ever make, which is anything but succinct.
But as with Ada and Haskell, you have compromise something for a strong type system.

\subsubsection{Redesign} is alsa a critical point in MC.
Because of the declarations needed, you will have te redo those entirely when rewriting or redisigning.
On the other hand, you can edit the implementation of a certain declaration any way you want, ass long as the input and output work the same way.

MC definitily has not been made for evolutionary programming.
It's more of a think before you type language.

\subsubsection{External Factors} are for nowquite

\end{multicols}

%\subsubsection{Data}
%\subsubsection{TypeFunc}
%\subsubsection{TypeAlias}
%\subsubsection{Func}
%\subsubsection{ArrowFunc}

%\subsubsection{Module}
% reference douwe for explained typesystem of module
%\subsubsection{Brackets}
%\subsubsection{Indentation}

%\subsubsection{Bar}
%\subsubsection{Rules}
%\subsubsection{Premise}
%\subsubsection{Arrows}
%\subsubsection{Small arrow}
%\subsubsection{Double arrow}
%\subsubsection{Priority arrow}
%\subsubsection{Comments}

%\subsection{Standard Library}
%\subsubsection{Boolean}
%\subsubsection{Monad}
%\subsubsection{TryableMonad}
%\subsubsection{Number}
%\subsubsection{Newnumber}
%\subsubsection{Match}
%\subsubsection{Prelude}
%\subsubsection{Record}

%\subsection{Basic Monads Library}
%\subsubsection{Either}
%\subsubsection{ID}
%\subsubsection{List}
%\subsubsection{Option}
%\subsubsection{Result}
%\subsubsection{State}

